\documentclass[CJK]{beamer}
\usepackage[slantfont, boldfont]{xeCJK}
\usepackage{ulem}
\usepackage{amsmath}
\usepackage{booktabs}
\usepackage{tikz}
\usepackage{colortbl}
\setCJKmainfont{SimSun}
\setCJKmonofont{SimSun}
\usetheme{PaloAlto}
\usecolortheme{rose}
\usefonttheme{professionalfonts}
\usepackage{CJK}
\usepackage{amsfonts}
\setlength{\parskip}{0.5\baselineskip}
\setlength{\parindent}{1em}

\begin{document}
\title{傻逼博弈杂讲}
\institute{ydc}
\date{\today}
\begin{frame}
\titlepage
\end{frame}
\begin{frame}
\frametitle{白书例题}
一个$1 \times n$的棋盘,有$X$和$.$,当棋盘上出现三个连续的$X$时游戏结束,两人轮流操作,每次能把一个$.$变成$X$,问先手必胜方案数

$n \le 1000$
\end{frame}
\begin{frame}
\frametitle{Codeforces 305E Playing with String}
刚开始你只有一个字符串

每次能选择一个有的字符串$s$,找到$i$,满足$s[i-1]=s[i+1]$,将其分裂成$3$个字符串$s[1 \cdots i-1],s[i],s[i+1 \cdots |s|]$

不能操作者负,求先手必胜的一个策略

初始字符串长度不超过$5000$
\end{frame}
\begin{frame}
\frametitle{HNOI 江南乐}
$n$堆石子,给定一个$F$,每次你能选择一个石子数大于等于$F$的石子堆,选择一个$m$,将其分成$m$堆,要求最大的一堆的个数和最小的一堆的个数差不能超过$1$,问先手是否必胜

每堆石子数不超过$100000$
\end{frame}
\begin{frame}
\frametitle{Codeforces 138D World of Darkraft}
有一个$n \times m$的棋盘,每个点上标记了$L,R,X$中的一个

每次能选择一个没有被攻击过的点$(i,j)$,从这个点开始发射线,射线形状为:

\begin{enumerate}
\item 若字符是$L$,向左下角和右上角发,遇到被攻击过的点就停下来
\item 若字符是$R$,向左上角和右下角发,遇到被攻击过的点就停下来
\item 若字符是$X$,向左小左上右下右上发,遇到被攻击过的点停下来
\end{enumerate}

问先手是否必胜,$n,m \le 20$
\end{frame}
\begin{frame}
\frametitle{HNOI 分裂游戏}
有$n$个瓶子,第$i$个瓶子里有$a_i$个豆子,每次能选择$i,j,k$,要求$i<j \le k$,第$i$个瓶子至少有一个豆子,随后从$i$里取出一个豆子,往$j,k$里各放入一个豆子(允许$j=k$),无法操作者输,问先手取胜策略数

$n \le 21,p_i \le 10000$
\end{frame}
\begin{frame}
\frametitle{POI 石子游戏}
有$n$堆石子,第$i$堆石子数大于等于第$i-1$堆,每次你能选择一堆石子移走拿走若干石子,但是要求拿走后仍然满足单调不降性质,问先手是否必胜
\end{frame}
\end{document}
